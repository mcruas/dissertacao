\documentclass{article}

\usepackage[utf8]{inputenc}


\title{}
\author{}
\date{\today}

\begin{document}

\maketitle

\section{B-splines}

Funções spline constituem uma base indicada para quando os dados possuem natureza não-periódica. 
Cada uma das funções $\phi_k$ é um polinômio, de uma ordem $m$ pré-estabelecida. 
Como regra de bolso, escolhe-se o grau do polinômio como duas unidades a mais que o número de vezes que a função será derivada.
A grande diferença entre uma base utilizando splines e uma base polinomial convencional é que cada uma das funções da base spline estará definida apenas para um subintervalo do intervalo $[t_0,t_1]$ no qual  deseja-se aproximar a função $f(t)$.
Os locais em que se dividem o intervalo total são pontos denominados de breakpoints.

A base B-spline é um conjunto específico de funções que são adequadas para fazer a aproximação desejada.
A literatura cita vantagens computacionais em se utilizar uma base B-splines em relação a outras, tanto no tempo de estimação dos coeficientes $\hat{c}_k$ como na disponibilidade de pacotes que o implementam em diversas linguagens e softwares diferentes.
Uma referência para aprofundar mais no estudo de splines pode ser encontrada em \citeonline{de_boor_2001}.

Graficamente, uma base B-spline é da seguinte forma:
-------inserir gráfico da base-----





\begin{equation}
y_i = x_{ij} + \varepsilon_i
\end{equation}

\section{}

\section{}

\end{document}
